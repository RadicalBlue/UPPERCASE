%%%%%%%%%%%%%%%%%%%%%%%%%%%%%%%%%%%%%%%%%%%%%%%%%%%%%%%%%%%%%%%%%%%%%%%%%%%%%%%%
% $Id: benv.tex 332 2015-06-30 08:59:57Z klugeflo $
%%%%%%%%%%%%%%%%%%%%%%%%%%%%%%%%%%%%%%%%%%%%%%%%%%%%%%%%%%%%%%%%%%%%%%%%%%%%%%%%

\section{Build Environment}
\label{s:benv}

%%%%%%%%%%%%%%%%%%%%%%%%%%%%%%%%%%%%%%%%%%%%%%%%%%%%%%%%%%%%%%%%%%%%%%%%%%%%%%%%

The whole build process of the \eb applications is managed by the
\code{build.py} script.
Concerning the embedded applications (\code{tg} and \code{ems}),
a number of \code{make} configuration files are used.
This section describes the configuration files and the variables
defined therein.

%%%%%%%%%%%%%%%%%%%%%%%%%%%%%%%%%%%%%%%%%%%%%%%%%%%%%%%%%%%%%%%%%%%%%%%%%%%%%%%%

\subsection{Platform-Independant Configuration}

The \eb \code{embedded} directory contains the following
platform-independant \code{make} configuration files:
\begin{description}
\item[\code{conf/build.mk}] manages the overall build process
\item[\code{\$APP/files.mk}] lists the (platform-independant)
  source files of application \code{\$APP}.
\end{description}

%%%%%%%%%%%%%%%%%%%%%%%%%%%%%%%%%%%%%%%%%%%%%%%%%%%%%%%%%%%%%%%%%%%%%%%%%%%%%%%%

\subsubsection{\code{build.mk}}

The whole build process of an embedded application is controlled by
the file \code{embedded/conf/build.mk}.
This file contains the rules to build the target ELF binary as well as
the relevant dependency and object files.
Therefore, it requires a number of variables that must be provided by
the platform-specific configuration files (see
sect.~\ref{ss:benv:pspec}):

The following variables must be defined in the actual \code{Makefile}
that includes the \code{build.mk} file (this is automatically done by
the Python scripts):
\begin{description}
\item[\code{BASE}] Path to the \code{embedded} directory, relative to
  the including \code{Makefile}.
\item[\code{ARCH}] Platform architecture name.
\item[\code{APP}] Name of the application that shall be built
  (currently, either \code{ems} or \code{tg}).
\item[\code{SUPP\_SRC}] Additional source code files that reside in
  the build directory (optional, needed for trace generator).
\end{description}
From the actual application, the following variables are imported:
\begin{description}
\item[\code{APP\_SRC}] lists the application's source files
\end{description}

Finally, a number of variables is required from the
platform-specific configuration files:
\begin{description}
\item[\code{HAL\_C\_SRC} and \code{HAL\_S\_SRC}]
  Generic \ac{hal} C and assembler sources.
\item[\code{HAL\_SUPP\_S\_SRC}]
  Supplementary assembler sources.
  These files are compiled, but not included in the link process.
  They should be linked by platform-specific linker flags,
  e.g. \verb+-msys-crt0='hal/crt0.o'+ on the NIOS platform.
  
\item[\code{HAP\_C\_SRC} and \code{HAP\_S\_SRC}]
  Application-specific \ac{hal} C and assembler sources.
\end{description}

%%%%%%%%%%%%%%%%%%%%%%%%%%%%%%%%%%%%%%%%%%%%%%%%%%%%%%%%%%%%%%%%%%%%%%%%%%%%%%%%

\subsubsection{\code{\$APP/files.mk}}

The \code{files.mk} file for each application defines the following
variables:
\begin{description}
\item[\code{THE\_APP\_SRC}] lists all source files of the
  application.
  Source files must reside directly inside the application directory,
  the build process does not support subdirectories.
\end{description}

%%%%%%%%%%%%%%%%%%%%%%%%%%%%%%%%%%%%%%%%%%%%%%%%%%%%%%%%%%%%%%%%%%%%%%%%%%%%%%%%

\subsection{Platform-Specific Configuration Files}
\label{ss:benv:pspec}

Inside each \code{arch/\$ARCH} directory, a number of \code{make}
configuration files provides platform-dependant build information:
\begin{description}
\item[\code{conf/toolchain.mk}] defines the toolchain variables
  (\code{CC}, \code{LD}).
  If you need to add some more flags, also do it in this files.
\item[\code{conf/upload.mk}] provides a rule to upload the
  \code{\$(TARGET).elf} to the hardware.
\item[\code{hal/files.mk}] lists all source files of the platform's
  generic \ac{hal}.
  Use \code{HAL\_C\_SRC} for C and \code{HAL\_S\_SRC} for assembler
  sources.
  Assembler files that are not directly linked must be referenced in
  \code{HAL\_SUPP\_S\_SRC}.
\item[\code{hal-\$APP/files.mk}] lists all source files of the
  application-specific \ac{hal}.
  Use \code{HAP\_C\_SRC} for C and \code{HAP\_S\_SRC} for assembler
  sources.
\end{description}


%%%%%%%%%%%%%%%%%%%%%%%%%%%%%%%%%%%%%%%%%%%%%%%%%%%%%%%%%%%%%%%%%%%%%%%%%%%%%%%%

\subsection{Building}
\label{ss:benv:build}

Use the \verb+build-ems.py+ resp. \verb+build-tg.py+ scripts in the
root directory.

%%%%%%%%%%%%%%%%%%%%%%%%%%%%%%%%%%%%%%%%%%%%%%%%%%%%%%%%%%%%%%%%%%%%%%%%%%%%%%%%

\subsection{The Application \code{Makefile}}

The actual Makefiles for the \code{tg} and \code{ems} applications are
generated by the Python build scripts.
They define some of the variables mentioned above and the include the
\verb+build.mk+ file that manges the build process.

%%%%%%%%%%%%%%%%%%%%%%%%%%%%%%%%%%%%%%%%%%%%%%%%%%%%%%%%%%%%%%%%%%%%%%%%%%%%%%%%

%%% Local Variables: 
%%% mode: latex
%%% TeX-master: porting
%%% TeX-PDF-mode: t
%%% End: 

%%%%%%%%%%%%%%%%%%%%%%%%%%%%%%%%%%%%%%%%%%%%%%%%%%%%%%%%%%%%%%%%%%%%%%%%%%%%%%%%
%<!-- Local IspellDict: english -->
