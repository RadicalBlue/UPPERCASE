%%%%%%%%%%%%%%%%%%%%%%%%%%%%%%%%%%%%%%%%%%%%%%%%%%%%%%%%%%%%%%%%%%%%%%%%%%%%%%%%
% $Id: implementation.tex 395 2015-10-02 14:22:13Z klugeflo $
%%%%%%%%%%%%%%%%%%%%%%%%%%%%%%%%%%%%%%%%%%%%%%%%%%%%%%%%%%%%%%%%%%%%%%%%%%%%%%%%

\section{Implementation}
\label{s:implementation}

%%%%%%%%%%%%%%%%%%%%%%%%%%%%%%%%%%%%%%%%%%%%%%%%%%%%%%%%%%%%%%%%%%%%%%%%%%%%%%%%


%%%%%%%%%%%%%%%%%%%%%%%%%%%%%%%%%%%%%%%%%%%%%%%%%%%%%%%%%%%%%%%%%%%%%%%%%%%%%%%%

\subsection{Preprocessor}

%%%%%%%%%%%%%%%%%%%%%%%%%%%%%%%%%%%%%%%%%%%%%%%%%%%%%%%%%%%%%%%%%%%%%%%%%%%%%%%%


Algorithm~\ref{a:preprocessor} shows the basic structure/functionality
of the preprocessor tool.
The following special cases are handled:
\begin{description}
  \item[driveaway] is detected if the \emph{startspeed} of an
    operation is 0 and the \emph{endspeed} is $\not=0$.
  \item[standstill] \emph{start-} and \emph{endspeed} of the operation
    are 0
  \item[clutch disengaged] is found if the currend speed is $\not=0$
    and gear is 0.
  \item[gearchange] if the gear of the current phase is different from
    the one of the previous phase.
\end{description}



\begin{algorithm}
  \caption{Preprocessing of driving cycle}
  \label{a:preprocessor}
  \begin{algorithmic}
    \Require OperationList contains all operation of the driving
    cycle, each operation is a tuple (acceleration, startspeed,
    endspeed, duration, gear)
    \Ensure PhaseList contains the corresponding phases for the generator
    \Procedure{Preprocess}{OperationList}
    %\State \textbf{declare} PhaseList
    \ForAll{op in OperationList}
    \If{driveaway}
    \Comment{Phases according to sect.~\ref{sss:model:driveaway}}
    \State PhaseList.enqueue(0, $t$) \Comment{$t$ according to eq.~\eqref{eq:driveaway:time}}
    \State PhaseList.enqueue($\alpha_N$, op.duration$-t$)
    \ElsIf{standstill}
    \State PhaseList.enqueue($\pm\alpha_I$, $t$) \Comment{$t$
      until $\omega_I$ reached}
    \State PhaseList.enqueue(0, op.duration$-t$)
    \ElsIf{clutch disengaged}
    \Comment{Phases according to sect.~\ref{sss:model:clutchdisen}}
    \State PhaseList.enqueue($\pm\alpha_I$, $t$) \Comment{until
      $\omega_I$ reached}
    \State PhaseList.enqueue(0, op.duration$-t$)
    \ElsIf{gearchange}
    \Comment{Phases according to sect.~\ref{sss:model:gearchange}}
    \State PhaseList.enqueue($\pm\alpha_I$, $\frac{1}{2}$op.duration)
    \State PhaseList.enqueue($\alpha_G$,  $\frac{1}{2}$op.duration)
    \Comment{$\alpha_G$ from eq.~\eqref{eq:omega:gear}}
    \Else\Comment{regular driving/acceleration/deceleration}
    \State PhaseList.enqueue(angular acceleration, op.duration)
    \EndIf
    \EndFor
    \EndProcedure
  \end{algorithmic}
\end{algorithm}

%%%%%%%%%%%%%%%%%%%%%%%%%%%%%%%%%%%%%%%%%%%%%%%%%%%%%%%%%%%%%%%%%%%%%%%%%%%%%%%%

\subsection{Generator}

%%%%%%%%%%%%%%%%%%%%%%%%%%%%%%%%%%%%%%%%%%%%%%%%%%%%%%%%%%%%%%%%%%%%%%%%%%%%%%%%

The generator has to implement two interrupt service routines, one for
the primary tooth (alg.~\ref{a:isr:primary}) and one for the secondary
tooth (alg.~\ref{a:isr:secondary}).
Most of the work is performed in the primary ISR when the signal is
deactivated.

\begin{algorithm}
  \caption{ISR for primary tooth}
  \label{a:isr:primary}
  \begin{algorithmic}
    \Procedure{PrimaryISR}{}
    \If{pin activate}
    \State set deactivation time
    \State \Return
    \Else
    \If{$k==0$}\Comment{Renormalise}
    \State $\omega_0\gets\omega(t)$
    \State $\varphi_0\gets 0$
    \State $t\gets 0$
    \EndIf
    \If{phase change pending}\Comment{Change Phase}
    \State $\alpha\gets\alpha_N$
    \EndIf
    \State calculate next $t_P$
    \State set primary activation time
    \If{$k==1$}\Comment{set secondary timer}
    \State calculate next $t_S$
    \State set secondary activation time
    \EndIf
    \EndIf
    \EndProcedure
  \end{algorithmic}
\end{algorithm}

\begin{algorithm}
  \caption{ISR for secondary tooth}
  \label{a:isr:secondary}
  \begin{algorithmic}
    \Procedure{SecondaryISR}{}
    \If{pin activate}
    \State set deactivation time
    \State \Return
    \Else
    \Comment{do nothing}
    \EndIf
    \EndProcedure
  \end{algorithmic}
\end{algorithm}

The following constants are additionally necessary:
\begin{description}
\item[TimeToTicks] convert time values from solution of equations to
  timer ticks
\end{description}


Notes on implementation experience:
\begin{itemize}
\item Times for primary OC must be set relatively to previous primary
\item Times for secondary OC should be set absolutely.
  Using relative numbers would increase complexity.
  Absolute number should be derived from previous primary time
  (trigger of calculation).
\end{itemize}


%%%%%%%%%%%%%%%%%%%%%%%%%%%%%%%%%%%%%%%%%%%%%%%%%%%%%%%%%%%%%%%%%%%%%%%%%%%%%%%%


%%%%%%%%%%%%%%%%%%%%%%%%%%%%%%%%%%%%%%%%%%%%%%%%%%%%%%%%%%%%%%%%%%%%%%%%%%%%%%%%


%%%%%%%%%%%%%%%%%%%%%%%%%%%%%%%%%%%%%%%%%%%%%%%%%%%%%%%%%%%%%%%%%%%%%%%%%%%%%%%%


%%%%%%%%%%%%%%%%%%%%%%%%%%%%%%%%%%%%%%%%%%%%%%%%%%%%%%%%%%%%%%%%%%%%%%%%%%%%%%%%


%%%%%%%%%%%%%%%%%%%%%%%%%%%%%%%%%%%%%%%%%%%%%%%%%%%%%%%%%%%%%%%%%%%%%%%%%%%%%%%%
%%% Local Variables: 
%%% mode: latex
%%% TeX-master: tg
%%% TeX-PDF-mode: t
%%% End: 
%%%%%%%%%%%%%%%%%%%%%%%%%%%%%%%%%%%%%%%%%%%%%%%%%%%%%%%%%%%%%%%%%%%%%%%%%%%%%%%%
%<!-- Local IspellDict: english -->
