%%%%%%%%%%%%%%%%%%%%%%%%%%%%%%%%%%%%%%%%%%%%%%%%%%%%%%%%%%%%%%%%%%%%%%%%%%%%%%%%
% $Id: model.tex 333 2015-06-30 13:00:39Z klugeflo $
%%%%%%%%%%%%%%%%%%%%%%%%%%%%%%%%%%%%%%%%%%%%%%%%%%%%%%%%%%%%%%%%%%%%%%%%%%%%%%%%

\section{Model}
\label{s:model}

%%%%%%%%%%%%%%%%%%%%%%%%%%%%%%%%%%%%%%%%%%%%%%%%%%%%%%%%%%%%%%%%%%%%%%%%%%%%%%%%


%%%%%%%%%%%%%%%%%%%%%%%%%%%%%%%%%%%%%%%%%%%%%%%%%%%%%%%%%%%%%%%%%%%%%%%%%%%%%%%%

\subsection{Model Assumptions}

%%%%%%%%%%%%%%%%%%%%%%%%%%%%%%%%%%%%%%%%%%%%%%%%%%%%%%%%%%%%%%%%%%%%%%%%%%%%%%%%

\begin{itemize}
\item Ideal driver and engine: engine speed adapts exactly according
  to driving cycle
\item no car speed is lost while clutch is disengaged for gear change
\item FreeEMS NipponDenso uses 24/2 sensor at cam shaft which has half
  the speed of the crank shaft. Simplify model by assuming 12/1 sensor
  mounted to crank shaft, amounts to same behaviour
\item engine yields constant acceleration if accelerated/decelerated
\end{itemize}

%%%%%%%%%%%%%%%%%%%%%%%%%%%%%%%%%%%%%%%%%%%%%%%%%%%%%%%%%%%%%%%%%%%%%%%%%%%%%%%%

\subsection{Rotary Sensor}
\label{ss:model:rotary}

%%%%%%%%%%%%%%%%%%%%%%%%%%%%%%%%%%%%%%%%%%%%%%%%%%%%%%%%%%%%%%%%%%%%%%%%%%%%%%%%

Figure~\ref{f:rs} depicts the behaviour of the rotary sensor.
The automaton uses the current angular speed $\varphi(t)$ of the crank
shaft as input.
Any time a primary or secondary tooth passes a sensor, a
corresponding signal $O_p$ or $O_S$ is emitted.

\begin{figure}
  \centering
  \begin{tikzpicture}[
      initial text={},%\begin{minipage}{1.5cm}$\omega_0:=\omega_L$\\$\varphi_0:=\alpha$\\$\alpha:=0$\end{minipage}},
      thick,node
      distance=1.5cm]
    \pgfsetcornersarced{\pgfpoint{1mm}{1mm}}
    \tikzstyle{state without output}=  [draw,every state,align=center]
    \node [state,initial] (running) {Running};

    \path %[transition]
    (running) edge [loop above] node [above] {$\varphi(t)\mbox{mod}\frac{1}{n_P}=0$/$O_P$} (running)
    (running) edge [loop below] node [below] {$\varphi(t)\mbox{mod}\frac{1}{n_S}=0$/$O_S$}(running)
    ;

    \node [above=of running] {
      \begin{tabular}{ll}
        \textbf{Input:} & $\varphi(t)\in\mathbb{R}^+$\\
        \textbf{Output:} & $O_P\in\{\mbox{absent},\mbox{present}\}$\\
        & $O_S\in \{\mbox{absent},\mbox{present}\}$
      \end{tabular}
    };
    
  \end{tikzpicture}
  \caption{Rotary Sensor}
  \label{f:rs}
\end{figure}


%%%%%%%%%%%%%%%%%%%%%%%%%%%%%%%%%%%%%%%%%%%%%%%%%%%%%%%%%%%%%%%%%%%%%%%%%%%%%%%%

\subsection{Crank Shaft}
\label{ss:model:crank}

%%%%%%%%%%%%%%%%%%%%%%%%%%%%%%%%%%%%%%%%%%%%%%%%%%%%%%%%%%%%%%%%%%%%%%%%%%%%%%%%

The behaviour of the crank shaft is depicted in figure~\ref{f:cs}.
The model develops the angular position $\varphi(t)$ and the angular
velocity $\omega(t)$.
Speed changes are managed through the angular acceleration
$\alpha$ which can be changed through the $\alpha_N$
input.
If a new $\alpha$ is set, the currently developed position and
speed are saved as new initial values, and the current time is stored
as new time offset.
Setting of $\alpha$ must ensure that the angular velocity
$\omega(t)$ never drops below $0$ (not contained in the model).

\begin{figure}
  \centering
  \begin{tikzpicture}[initial
      text={\begin{minipage}{1.5cm}$\omega_0:=\omega_I$\\$\varphi_0:=\alpha$\\$\alpha:=0$\\$t_0=t$\end{minipage}},thick,node
      distance=2.0cm]
    \pgfsetcornersarced{\pgfpoint{1mm}{1mm}}
    \tikzstyle{state without output}=  [draw,every state,align=center]
    \node [state,initial] (running) {Running\\
      $\varphi(t)=\frac{1}{2}\alpha(t-t_0)^2+\omega_0(t-t_0)+\varphi_0$\\
      $\omega(t)=\alpha(t-t_0)+\omega_0$
    };

    \path %[transition]
    (running) edge [loop above] node [above,align=center] {
      $\alpha_N$/-\\
      $\alpha:=\alpha_N, \omega_0:=\omega(t),
      \varphi_0:=\varphi(t),t_0:=t$
    } (running);

    \node [above=of running] {
      \begin{tabular}{ll}
        \textbf{Input:} & $\alpha_N\in\mathbb{R}\cup\{\mbox{absent}\}$\\
        \textbf{Output:} & $\omega(t)\in\mathbb{R}$\\
        & $\varphi(t)\in\mathbb{R}^+$\\
        \textbf{Variables:} & $\alpha\in\mathbb{R}$\\
        & $\omega_0,\varphi_0,t_0\in\mathbb{R}^+$\\
      \end{tabular}
    };
  \end{tikzpicture}
  \caption{Crank shaft behaviour}
  \label{f:cs}
\end{figure}

%%%%%%%%%%%%%%%%%%%%%%%%%%%%%%%%%%%%%%%%%%%%%%%%%%%%%%%%%%%%%%%%%%%%%%%%%%%%%%%%

\subsection{Determination of $\alpha$}
\label{ss:model:dotomega}

%%%%%%%%%%%%%%%%%%%%%%%%%%%%%%%%%%%%%%%%%%%%%%%%%%%%%%%%%%%%%%%%%%%%%%%%%%%%%%%%

Only under certain circumstances, accelerations $a$ of the car from a
driving cycle can be translated directly into angular accelerations of
the crank shaft.
There are a number of special situations, that also require some more
work.
Basically, we can describe the behaviour of the crank shaft as
consisting of phases with constant angular acceleration.
Each phase is a tuple $(\alpha,d)$ where $\alpha$ is the acceleration
of this phase, and $d$ is its duration.

%%%%%%%%%%%%%%%%%%%%%%%%%%%%%%%%%%%%%%%%%%%%%%%%%%%%%%%%%%%%%%%%%%%%%%%%%%%%%%%%

\subsubsection{Regular Operation}

A regular operation is on hand, if (1) the acceleration during an
operation is constant, and (2) there is no gear change during the
operation.
These condition is fulfilled by the following operations of a driving
cycle: \emph{idling}, \emph{steady speed}, \emph{deceleration}.
Additionally, if an \emph{acceleration} operation has a start speed
$\not=0$, it is also considered as a regular operation.

\paragraph{Idling}
\begin{equation}
  \label{eq:alpha:idling}
  \alpha = \alpha_I
\end{equation}


\paragraph{Steady Speed}
\begin{equation}
  \label{eq:alpha:steady}
  \alpha = 0
\end{equation}

\paragraph{Acceleration and Deceleration}
\begin{equation}
  \label{eq:alpha:ac-deceleration}
  \alpha = r_ag_i\frac{a}{c_w}\frac{1000\mbox{mm}}{\mbox{m}}
\end{equation}


%%%%%%%%%%%%%%%%%%%%%%%%%%%%%%%%%%%%%%%%%%%%%%%%%%%%%%%%%%%%%%%%%%%%%%%%%%%%%%%%

\subsubsection{Driveaway from Standstill}
\label{sss:model:driveaway}

Actual behaviour: Driver slowly releases clutch, may press
accelerator pedal.

Modeling: Simplify! Assume engine speed is constant (idle speed) until
clutch closed, then increases according to acceleration. Use
acceleration to calculate time until clutch closed (where engine idle speed
matches car speed).

relationship car speed $\leftrightarrow$ engine speed:
\begin{equation}
  \omega(t) = r_ag_i\frac{v(t)}{\frac{c_w}{1000}\frac{\mbox{m}}{\mbox{mm}}}
\end{equation}
with
\begin{equation}
  v(t) = at + v_0
\end{equation}

Now we need to find the time $t_I$ with $\omega(t_I) = \omega_I$.

Thus solve for $t$:
\begin{equation}
  r_ag_i\frac{at_I + v_0}{\frac{c_w}{1000}\frac{m}{mm}} = \omega_I
\end{equation}
with $v_0=0$ due to standstill, we get:
\begin{equation}
  \label{eq:driveaway:time}
  t_I = \frac{\omega_I}{r_ag_ia}\frac{c_w}{1000}\frac{m}{mm}
\end{equation}

Thus, we have a first phase of duration $t_I$ according to
eq.~\ref{eq:driveaway:time} where $\alpha=0$ and $\omega(t)=\omega_I$.
In in the second phase, $\alpha$ is calculated according to
eq.~\ref{eq:alpha:ac-deceleration}.

%%%%%%%%%%%%%%%%%%%%%%%%%%%%%%%%%%%%%%%%%%%%%%%%%%%%%%%%%%%%%%%%%%%%%%%%%%%%%%%%

\subsubsection{Gear Change}
\label{sss:model:gearchange}

Actual behaviour: driver presses clutch and releases accelarator pedal,
changes gear, releases clutch and presses accelerator.

Modeling: Split gear change operation in two phases with equal
duration.
Let $d_g$ be the total duration of the gear change operation.
Then each phase has a duration of $\frac{d_g}{2}$.
\begin{enumerate}
\item \textbf{Disengagement of clutch:}
  Assume that clutch and accelerator pedal are disengaged instantly at
  the beginning of this phase.
  Then, $\omega$ changes with $\alpha_I$ until $\omega_I$ reached or
  end of phase is reached.
  Depending on the initial angular velocity $\omega_0$, the angular acceleration
  must be chosen as:
  \begin{equation}
    \label{eq:alpha:gear:diseng}
    \alpha_g = \left\{
      \begin{array}{c@{\;:\;}l}
        -\alpha_I&\omega_0>\omega_I\\
        0 & \omega_0=\omega_I\\
        \alpha_I&\omega_0<\omega_I
      \end{array}
      \right.
  \end{equation}
  Angular velocity develops as:
  \begin{equation}
    \label{eq:omega}
    \omega(t) = \omega_0 + \alpha_g t
  \end{equation}
  Assume that phase starts at $t=0$. Distinguish \underline{two} cases:
  \begin{enumerate}
  \item If there exists a $t_I<\frac{d_g}{2}$ such that $\omega(t_I)=\omega_I$,
    introduce two phases $(\alpha_g,t_I)$, $(0,\frac{d_g}{2}-t_I)$.
  \item If no such $t_I$ exists, introduce \underline{one} phase $(\alpha,\frac{d_g}{2})$.
  \end{enumerate}
\item \textbf{Engagement of clutch with new gear:}
  Assume that the gear is changed at the beginning of this phase, and
  the clutch is finally closed exactly at the end of this phase.
  The aim is to develop $\omega(t)$ in such a manner that
  $\omega(d_g)$ matches the new rotation speed at the end of the
  phase.
  To simplify calculations, we renormalise the time base to $t=0$ at
  the beginning of the phase, and let $\omega_0$ be the angular
  velocity at the end of the previous phase.
  Let $v$ be the current speed of the car ($v_e$ of previous and $v_s$
  of next phase).
  Then, the target angular speed $\omega(\frac{d_g}{2})$ is
  \begin{equation}
    \omega(\frac{d_g}{2}) =
    r_ag_i\frac{v}{\frac{c_w}{1000}\frac{\mbox{m}}{\mbox{mm}}}
  \end{equation}
  Then, $\alpha_G$ is calculated as
  \begin{equation}
    \label{eq:omega:gear}
    \alpha_G = \frac{2}{d_g}\left(\omega(\frac{d_g}{2}) -
    \omega_0\right)
    =
    \frac{2}{d_g}\left(r_ag_i\frac{v}{\frac{c_w}{1000}\frac{\mbox{m}}{\mbox{mm}}}
    -\omega_0\right)
  \end{equation}

  %% gear change at beginning, $\omega$ must change such that it
  %% matches new (rotation) speed when clutch closed at end of phase.
  %% Need to find $\alpha_G$ such that
  %% \begin{equation}
  %%   \omega(t) = \omega_0 + \alpha_Gt
  %% \end{equation}
  %% with
  %% \begin{equation}
  %%   \omega(t) = \frac{v(t)}{\frac{d_w}{1000}\frac{m}{mm}}
  %% \end{equation}
  %% with $t$ end of phase. Solving for $\alpha_G$:
  %% \begin{equation}
  %%   \label{eq:omega:gear}
  %%   \alpha_G = \frac{1}{t}\left(\frac{v(t)}{\frac{d_w}{1000}\frac{m}{mm}} - \omega_0\right)
  %% \end{equation}
\end{enumerate}

Further assume that no speed is lost while clutch is disengaged.

%%%%%%%%%%%%%%%%%%%%%%%%%%%%%%%%%%%%%%%%%%%%%%%%%%%%%%%%%%%%%%%%%%%%%%%%%%%%%%%%

\subsubsection{Deceleration with Clutch Disengaged}
\label{sss:model:clutchdisen}

When the car is decelerated with the clutch being disengaged, the
engine speed should approximate $\omega_I$.
Let $d_d$ be the total duration of this operation.
We model this operation with two phases:
\begin{enumerate}
\item\textbf{Approximation:}
  If $\omega(t)=\omega_I$, this phase is skipped.
  Else, the engine speed has to drop or rise to $\omega_I$.
  Therefore, $\alpha_d$ is chosen according to
  eq.~\ref{eq:alpha:gear:diseng}.
  Again, assume that this phase starts at $t=0$.
  Then, the phase ends at $t_a$ with:
  \begin{equation}
    \omega(t_a) = \omega_I
  \end{equation}
  Combine with eq.~\ref{eq:omega} and solve for $t_a$ and add phase
  $(\alpha_d,\min(t_a,d_d))$.
  
\item\textbf{Idling:}
  If $t_a\geq d_d$, this phase is skipped.
  Else, the engine speed stays constant.
  We add a phase $(0, d_d-t_a)$.
\end{enumerate}

%%%%%%%%%%%%%%%%%%%%%%%%%%%%%%%%%%%%%%%%%%%%%%%%%%%%%%%%%%%%%%%%%%%%%%%%%%%%%%%%


%%%%%%%%%%%%%%%%%%%%%%%%%%%%%%%%%%%%%%%%%%%%%%%%%%%%%%%%%%%%%%%%%%%%%%%%%%%%%%%%


%%%%%%%%%%%%%%%%%%%%%%%%%%%%%%%%%%%%%%%%%%%%%%%%%%%%%%%%%%%%%%%%%%%%%%%%%%%%%%%%


%%%%%%%%%%%%%%%%%%%%%%%%%%%%%%%%%%%%%%%%%%%%%%%%%%%%%%%%%%%%%%%%%%%%%%%%%%%%%%%%


%%%%%%%%%%%%%%%%%%%%%%%%%%%%%%%%%%%%%%%%%%%%%%%%%%%%%%%%%%%%%%%%%%%%%%%%%%%%%%%%


%%%%%%%%%%%%%%%%%%%%%%%%%%%%%%%%%%%%%%%%%%%%%%%%%%%%%%%%%%%%%%%%%%%%%%%%%%%%%%%%


%%%%%%%%%%%%%%%%%%%%%%%%%%%%%%%%%%%%%%%%%%%%%%%%%%%%%%%%%%%%%%%%%%%%%%%%%%%%%%%%


%%%%%%%%%%%%%%%%%%%%%%%%%%%%%%%%%%%%%%%%%%%%%%%%%%%%%%%%%%%%%%%%%%%%%%%%%%%%%%%%


%%%%%%%%%%%%%%%%%%%%%%%%%%%%%%%%%%%%%%%%%%%%%%%%%%%%%%%%%%%%%%%%%%%%%%%%%%%%%%%%
%%% Local Variables: 
%%% mode: latex
%%% TeX-master: tg
%%% TeX-PDF-mode: t
%%% End: 
%%%%%%%%%%%%%%%%%%%%%%%%%%%%%%%%%%%%%%%%%%%%%%%%%%%%%%%%%%%%%%%%%%%%%%%%%%%%%%%%
%<!-- Local IspellDict: english -->
